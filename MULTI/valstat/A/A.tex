\documentclass[12pt]{article}

\usepackage{setspace}
\onehalfspace

\usepackage{amsmath}
\usepackage{moodle}


\begin{document}

\begin{quiz}{valogat1}
\begin{multi}{botmax}
Egy $ 6 $ méter hosszú cérnát egy ollóval véletlenszerűen elvágunk.
Legyen $ Y $ a keletkezett részek hosszának maximuma.
Ekkor $ Y $ eloszlásfüggvénye: (ami a jelzett intervallumtól jobbra 1,
balra pedig 0 értéket vesz fel)
\item* $ F(x)=\frac{x-3}{3},\:\: 3<x<6 $
\item $ F(x)=\frac{x}{3},\:\: 3<x<6 $
\item $ F(x)=\frac{x-3}{6},\:\: 0<x<6 $
\item $ F(x)=\frac{x}{6},\:\: 0<x<6 $
\end{multi}
\begin{multi}{botmaxvarhato}
Egy $ 10 $ méter hosszú cérnát egy ollóval véletlenszerűen elvágunk.
Legyen $ Y $ a keletkezett részek hosszának maximuma.
Ekkor az $ Y $ várható értéke:
\item* $ \frac{15}{2} $
\item $ 5 $
\item $ 10 $
\item $ \frac{5}{4} $
\end{multi}
\begin{multi}{cernaminvarhato}
Egy $ 10 $ méter hosszú cérnát egy ollóval véletlenszerűen elvágunk.
Legyen $ Y $ a keletkezett részek hosszának minimuma.
Ekkor $\mathrm{E}(Y)=$
\item* $ \frac{5}{2} $
\item $ \frac{12}{5} $
\item $ \frac{27}{10} $
\item $ \frac{29}{10} $
\end{multi}
\begin{multi}{cernaminval}
Egy $ 9 $ méter hosszú cérnát egy ollóval véletlenszerűen elvágunk.
Legyen $ Y $ a keletkezett részek hosszának minimuma.
Ekkor $ \mathrm{P}(\frac{1}{20}<Y<\frac{3}{5}) $=
\item* $ \frac{11}{90} $
\item $ \frac{7}{30} $
\item $ \frac{31}{90} $
\item $ \frac{17}{30} $
\end{multi}
\begin{multi}{botmaxval}
Egy $ 8 $ méter hosszú botot egy fűrésszel elvágunk egy véletlenszerűen választtott
helyen.
Legyen $ Y $ a keletkezett részek hosszának maximuma.
Ekkor $ \mathrm{P}(5<Y<7) $=
\item* $ \frac{1}{2} $
\item $ \frac{1}{4} $
\item $ \frac{1}{8} $
\item $ \frac{3}{4} $
\end{multi}
\begin{multi}{cernamineo}
Egy $ 8 $ méter hosszú cérnát egy ollóval véletlenszerűen elvágunk.
Legyen $ Y $ a keletkezett részek hosszának minimuma.
Ekkor $ Y $ eloszlásfüggvényének nem-konstans része:
\item* $ \frac{x}{4},\:\: 0<x<4 $
\item $ \frac{x-8}{4},\:\: 4<x<8 $
\item $ \frac{x}{8},\:\: 0<x<8 $
\item $ \frac{x-8}{8},\:\: 0<x<4 $
\end{multi}
\end{quiz}
\end{document}
